% \iffalse meta-comment
%% lxir.dtx
%% part of lxir - a tool to convert latex documents to xml.
%% Copyright 2007 EDP-Sciences
%
% This work may be distributed and/or modified under the
% conditions of the LaTeX Project Public License, either version 1.3
% of this license or (at your option) any later version.
% The latest version of this license is in
%   http://www.latex-project.org/lppl.txt
% and version 1.3 or later is part of all distributions of LaTeX
% version 2005/12/01 or later.
%
% This work has the LPPL maintenance status `maintained'.
% 
% The Current Maintainer of this work is EDP-Sciences.
%
% This work consists of the files lxir.dtx, lxir.ins and the
% derived files lxir.sty and lxirtag.sty.
%
% Part of this file has been pickup from or inspired by
% Hermes macro definitions for (AMS)LaTEX version 0.9.10, 2006-07-27
% author Romeo Anghelache http://romeo.roua.org
% Other parts come from latex.ltx (LaTeX3 project) distributed with TeXLive 2007
%
% modifs a mettre dans svn : 
% -------------------------
% 
% \fi
%% \CharacterTable
%%  {Upper-case    \A\B\C\D\E\F\G\H\I\J\K\L\M\N\O\P\Q\R\S\T\U\V\W\X\Y\Z
%%   Lower-case    \a\b\c\d\e\f\g\h\i\j\k\l\m\n\o\p\q\r\s\t\u\v\w\x\y\z
%%   Digits        \0\1\2\3\4\5\6\7\8\9
%%   Exclamation   \!     Double quote  \"     Hash (number) \#
%%   Dollar        \$     Percent       \%     Ampersand     \&
%%   Acute accent  \'     Left paren    \(     Right paren   \)
%%   Asterisk      \*     Plus          \+     Comma         \,
%%   Minus         \-     Point         \.     Solidus       \/
%%   Colon         \:     Semicolon     \;     Less than     \<
%%   Equals        \=     Greater than  \>     Question mark \?
%%   Commercial at \@     Left bracket  \[     Backslash     \\
%%   Right bracket \]     Circumflex    \^     Underscore    \_
%%   Grave accent  \`     Left brace    \{     Vertical bar  \|
%%   Right brace   \}     Tilde         \~}
% \GetFileInfo{lxir.dtx}
% \title{\Package{lxir} and \Package{lxirtag}: Tagging of \LaTeX{} for \LXir{}}
% \author{Jean-Paul Jorda \and Xavier Trochu (EDP Sciences)}
% \maketitle
%    \begin{macrocode}
%<*driver>
\NeedsTeXFormat{LaTeX2e}[1995/12/01]
\documentclass{ltxdoc}
\usepackage{lxirdoc}
\EnableCrossrefs
\CodelineIndex
\RecordChanges
\begin{document}
\DocInput{lxir.dtx}
\end{document}
%</driver>
%    \end{macrocode}
% \section{How to use these packages ?}
%
%\subsection{The \LXir{} project}
% The packages \Package{lxir} and \Package{lxirtag} are part of \LXir{}, a \LaTeX{}\RA XML converter.
% You don't need them if you don't intent to use \LXir{}.
% Please refert to the documentation of the \LXir{} project on \verb|www.latex-lxir.org|.
% \subsection{Using of \Package{lxir} for \LaTeX{}\RA XML conversion with \LXir{}}
% The package \Package{lxir} is must be loaded \emph{before} the class call:
% \begin{verbatim}
% \RequirePackage{lxir}
% \documentclass{article}
% ...
% \end{verbatim}
% You can then run \Program{latex}, then \Program{lxir} to got an XML file:
% \begin{verbatim}
% $> latex mydoc.tex
% $> lxir mydoc.dvi
% \end{verbatim}
% The package \Package{lxir} loads \Package{lxirtag}, which define a set of commands whose purpose is to tag the dvi 
% with ``\cs{special}'' instructions. The content of these instructions are recognised by the \File{lxir}
% program and transformed to  XML tags. 
% The package \Package{lxir}  also aims to:
% \begin{itemize}
% \item set some \TeX{} parameters for word and paragraph tagging;
% \item overload the commands and environments defined in \File{latex.ltx}
% \end{itemize}
% One of the effect of the \File{latex.ltx} command overload is to enable a 
% mechanism for classes and packages tagging.
% \subsection{Using of \Package{lxirtag} to write ``\LXir{} compliant'' classes or a packages}
% If you want to write a class or a package ready to use with \LXir{}, you can include \Module{lxirtag}
% and use the macros provided for tagging:
% \begin{verbatim}
% \RequirePackage{lxirtag}
% \end{verbatim}
%
%    \begin{macrocode}
\NeedsTeXFormat{LaTeX2e}[1994/05/13]
%<*lxir>
\ProvidesPackage{lxir}[2007/06/06 v0.1b LXir tagging package]
\newif\ifverbatimmath
\verbatimmathfalse
\DeclareOption{verbatimmath}{\verbatimmathtrue}
\ExecuteOptions{}
\ProcessOptions
\RequirePackage{lxirtag}[2007/06/06]
%</lxir>
%<lxirtag>\ProvidesPackage{lxirtag}[2007/06/06 v0.1b LXir tagging package]
%    \end{macrocode}
% \section{\TeX{} Parameters Setting}
% Several \TeX{} parameters are set to ease to tagging and 
% to prevent hyphenation. A potential problem is that
% these parameters may be reset in packages. 
% \subsection{Word,Page, Paragraph, hyphenation}
% The spaces between words are tagged to 1sp so that word  breaks can be
% clearly identified in the DVI:
%    \begin{macrocode}
%<*lxir>
%\parindent=1sp
\AtBeginDocument{\spaceskip=1sp\xspaceskip=1sp}
%    \end{macrocode}
% and we prevent hyphenation: 
%    \begin{macrocode}
\AtBeginDocument{\pretolerance=10000}
%    \end{macrocode}
% 
% The italic correction is set to null,  to preserved the 1sp space between words:
%    \begin{macrocode}
\AtBeginDocument{\let\@@italiccorr=\relax}
%    \end{macrocode}
% We also want to forbid line break breaks in math:
%    \begin{macrocode}
\AtBeginDocument{\binoppenalty=10000\relpenalty=10000}
%    \end{macrocode}
% Paragraph width is set to almost the maximum: 
%    \begin{macrocode}
\AtBeginDocument{\hsize=0.99\maxdimen}
\setlength\textwidth{0.99\maxdimen} % need to be set again to this value in classes?
%    \end{macrocode}
% Page height is set to almost the maximum: 
%    \begin{macrocode}
\AtBeginDocument{\vsize=0.99\maxdimen}
%    \end{macrocode}
% We ``neutralize'' the previously used commands and dimensions.  
%    \begin{macrocode}
\AtBeginDocument{%
\newskip\junkskip
\let\spaceskip=\junkskip
\let\xspaceskip=\junkskip
\newdimen\junkdim
\let\vsize=\junkdim
\let\hsize=\junkdim
}
%    \end{macrocode}
% And we set the pagestyle to empty, in order to avoid page numbers and so on.
% Then we prevent savagely the use of \cmd{\pagestyle} and \cmd{\thispagestyle}.
%    \begin{macrocode}
\AtBeginDocument{\pagestyle{empty}\def\pagestyle#1{\relax}\def\thispagestyle#1{\relax}}
%    \begin{macrocode}
% \section{Outputing information}
%    \begin{macrocode}
\newcommand{\LXir@Info}[1]{%
  \GenericWarning{}{[LXir]   #1}
}
%    \end{macrocode}
% \section{Macros Used for the Tagging}
% \label{tagging}
%\subsection{``\cmd{\special}'' tagging  for \Program{lxir}}
% We use |\special{}| for the tagging of commands and environments in order
% to produce element nodes in the resulting XML. Both empty and non empty elements can 
% be generate.
%
%\subsubsection{Non Empty nodes}
%\label{sec:nonemptynode}
% \begin{verbatim}
% \special{:: tag lxir begin(command){attr1=val1}{attr2=val2}}{...}
% some text
% \special{:: tag lxir end(command){attr1=val1}{attr2=val2}}{...}
% \end{verbatim}
% will result in a node:
% \begin{verbatim}
% <node type="command" attr1="val1" attr2="val2">
%   <text font="14" h="7488848" v="6121060">some text</text>
% </node>
% \end{verbatim}
%\subsubsection{Empty nodes}
%\label{sec:emptynode}
% \begin{verbatim}
% \special{:: tag lxir empty(command){attr1=val1}{attr2=val2}}{...}
% \end{verbatim}
% will result in an empty node:
% \begin{verbatim}
% <node type="command" attr1="val1" attr2="val2"/>
% \end{verbatim}
% \subsubsection{Entities}
% In most cases \LXir{} can generate unicode characters
% by reading and interpreting fonts information in the dvi. In some case
% case, this information in not present  and
% must be tagged. This is not done by a special but by inserting 
% a string of the form [entity!\#xN!] where N is the hexadecimal unicode number.
% (e.g. [entity!\#5F!] for underscore).
%\subsection{Macros defined for the tagging}
% A set of macros are defined to ease the tagging and to provide an unique 
% id attribute for each node:
% \pagebreak
% The macro |\xBegin|, together with |\xEnd|,  are used to  create a node 
% containing other nodes:
% \begin{verbatim}
% \xBegin{myTag}something\xEnd{myTag}
% \end{verbatim}
% will be output as
% \begin{verbatim}
% <node type="myTag" id="0">
%   <text font="14" h="7488848" v="6121060">something</text>
% </node>
% \end{verbatim}
% in the XML.
% These commands are used to produce empty nodes, with attributes: \\ 
% |\xEmpty{myTag}| will be output  |<node type="myTag" id="0">|, \\
% |\xEmptyA{myTag}{myatrr=a}| will be output  |<node type="myTag" id="0" myatrr="a">|,\\ 
% |\xEmptyA{myTag}{myatrr1=a}{myatrr2=b}| will be output \\ |<node type="myTag" id="0" myatrr1="a" myatrr2="b">|,\\
% and so on .
% \par
% |\xT| is used to tag the accented and ``special'' characters.
%    \begin{macrocode}
%</lxir>
%<*lxirtag>
\newcount\xTagcount
\xTagcount=0
\def\xNextTag{\global\advance\xTagcount by 1}
\newif\ift@gDebug
\t@gDebugfalse
\newif\ift@gOn
\@ifpackageloaded{lxir}{\t@gOntrue}{\t@gOnfalse}
\def\xBegin#1{\ift@gOn\ift@gDebug\message{BEGIN: #1^^J}\fi\special{::tag lxir begin(#1){id=\the\xTagcount}}\xNextTag\relax\fi}
\def\xEnd#1{\ift@gOn\ift@gDebug\message{END: #1^^J}\fi\special{::tag lxir end(#1){id=\the\xTagcount}}\xNextTag\relax\fi}
\def\xEmpty#1{\ift@gOn\ift@gDebug\message{EMPTY: #1^^J}\fi\special{::tag lxir empty(#1){id=\the\xTagcount}}\xNextTag\relax\fi}
\def\xEmptyA#1#2{\ift@gOn\ift@gDebug\message{EMPTY: #1 {#2}^^J}\fi\special{::tag lxir empty(#1){id=\the\xTagcount} {#2}}\xNextTag\relax\fi}
\def\xEmptyAA#1#2#3{\ift@gOn\ift@gDebug\message{EMPTY: #1 {#2}{#3}^^J}\fi\special{::tag lxir empty(#1){id=\the\xTagcount} {#2}{#3}}\xNextTag\relax\fi}
\def\xEmptyAAA#1#2#3#4{\ift@gOn\ift@gDebug\message{EMPTY: #1 {#2}{#3}{#4}^^J}\fi\special{::tag lxir empty(#1){id=\the\xTagcount} {#2}{#3}{#4}}\xNextTag\relax\fi}
\def\xEmptyAAAA#1#2#3#4#5{\ift@gOn\ift@gDebug\message{EMPTY: #1 {#2}{#3}{#4}{#5}^^J}\fi\special{::tag lxir empty(#1){id=\the\xTagcount} {#2}{#3}{#4}{#5}}\xNextTag\relax\fi}
\def\xEmptyAAAAA#1#2#3#4#5#6{\ift@gOn\ift@gDebug\message{EMPTY: #1 {#2}{#3}{#4}{#5}{#6}^^J}\fi\special{::tag lxir empty(#1){id=\the\xTagcount} {#2}{#3}{#4}{#5}{#6}}\xNextTag\relax\fi}
\def\xEmptyAAAAAA#1#2#3#4#5#6#7{\ift@gOn\ift@gDebug\message{EMPTY: #1 {#2}{#3}{#4}{#5}{#6}{#7}^^J}\fi\special{::tag lxir empty(#1){id=\the\xTagcount} {#2}{#3}{#4}{#5}{#6}{#7}}\xNextTag\relax\fi}
%    \end{macrocode}
% Element attribute key="value" in the output XML will be add if |\xEmptyA...{params}{key=value}...| follow the element. The following 
% commands are short cuts for the feature. By example : 
% \begin{verbatim}
% \xBegin{myCommand}\xAttrA{myKey1=value1}{myKey2=value2}%
%  ...
% \xEnd{myCommand}%
% \end{verbatim}
% will be merged by lxir as 
% \begin{verbatim}
% <span class="myCommand" myKey1="value1" myKey2="value2">
% ...
% </span>
% \end{verbatim}
%    \begin{macrocode}
\def\xAttr#1{\ift@gOn\ift@gDebug\message{ATTR: {#1}^^J}\fi\special{::tag lxir empty(params){id=\the\xTagcount} {#1}}\xNextTag\relax\fi}
\def\xAttrA#1#2{\ift@gOn\ift@gDebug\message{ATTR: {#1}{#2}^^J}\fi\special{::tag lxir empty(params){id=\the\xTagcount} {#1}{#2}}\xNextTag\relax\fi}
\def\xAttrAA#1#2#3{\ift@gOn\ift@gDebug\message{ATTR: {#1}{#2}{#3}^^J}\fi\special{::tag lxir empty(params){id=\the\xTagcount} {#1}{#2}{#3}}\xNextTag\relax\fi}
\def\xAttrAAA#1#2#3#4{\ift@gOn\ift@gDebug\message{ATTR: {#1}{#2}{#3}{#4}^^J}\fi\special{::tag lxir empty(params){id=\the\xTagcount} {#1}{#2}{#3}{#4}}\xNextTag\relax\fi}
\def\xAttrAAAA#1#2#3#4#5{\ift@gOn\ift@gDebug\message{ATTR: {#1}{#2}{#3}{#4}{#5}^^J}\fi\special{::tag lxir empty(params){id=\the\xTagcount} {#1}{#2}{#3}{#4}{#5}}\xNextTag\relax\fi}
\def\xAttrAAAAA#1#2#3#4#5#6{\ift@gOn\ift@gDebug\message{ATTR: {#1}{#2}{#3}{#4}{#5}{#6}^^J}\fi\special{::tag lxir empty(params){id=\the\xTagcount} {#1}{#2}{#3}{#4}{#5}{#6}}\xNextTag\relax\fi}
%</lxirtag>
%<*lxir>\t@gOntrue
%    \end{macrocode}
%\section{Accents and miscaneleous tagging}
%\label{sec:accent}
%\subsection{Miscaleneous}
%    \begin{macrocode}
\def\TeX{\xEmpty{TeX}}
\let\t@gdoubleantislash=\\
\def\\{\xEmpty{br}\t@gdoubleantislash}
%  \LaTeX
\def\LaTeXe\xEmpty{LaTeXe}
\def\LaTeX{\xEmpty{LaTeX}}
%    \end{macrocode}
% Paragraph separation: 
%    \begin{macrocode}
\let\t@gpar=\par
\def\par{\xEmpty{parsep}\t@gpar}
%    \end{macrocode}
% MATH
\def\mathopt@g#1{\xEmptyAA{#1}{mathtype=op}{mathcontent=#1}}
\def\log{\mathop{\mathopt@g{log}}\nolimits}
\def\lg{\mathop{\mathopt@g{lg}}\nolimits}
\def\ln{\mathop{\mathopt@g{ln}}\nolimits}
\def\lim{\mathop{\mathopt@g{lim}}}
\def\limsup{\mathop{\mathopt@g{limsup}}}
\def\liminf{\mathop{\mathopt@g{liminf}}}
\def\sin{\mathop{\mathopt@g{sin}}\nolimits}
\def\arcsin{\mathop{\mathopt@g{arcsin}}\nolimits}
\def\sinh{\mathop{\mathopt@g{sinh}}\nolimits}
\def\cos{\mathop{\mathopt@g{cos}}\nolimits}
\def\arccos{\mathop{\mathopt@g{arccos}}\nolimits}
\def\cosh{\mathop{\mathopt@g{cosh}}\nolimits}
\def\tan{\mathop{\mathopt@g{tan}}\nolimits}
\def\arctan{\mathop{\mathopt@g{arctan}}\nolimits}
\def\tanh{\mathop{\mathopt@g{tanh}}\nolimits}
\def\cot{\mathop{\mathopt@g{cot}}\nolimits}
\def\coth{\mathop{\mathopt@g{coth}}\nolimits}
\def\sec{\mathop{\mathopt@g{sec}}\nolimits}
\def\csc{\mathop{\mathopt@g{csc}}\nolimits}
\def\max{\mathop{\mathopt@g{max}}}
\def\min{\mathop{\mathopt@g{min}}}
\def\sup{\mathop{\mathopt@g{sup}}}
\def\inf{\mathop{\mathopt@g{inf}}}
\def\arg{\mathop{\mathopt@g{arg}}\nolimits}
\def\ker{\mathop{\mathopt@g{ker}}\nolimits}
\def\dim{\mathop{\mathopt@g{dim}}\nolimits}
\def\hom{\mathop{\mathopt@g{hom}}\nolimits}
\def\det{\mathop{\mathopt@g{det}}}
\def\exp{\mathop{\mathopt@g{exp}}\nolimits}
\def\Pr{\mathop{\mathopt@g{Pr}}}
\def\gcd{\mathop{\mathopt@g{gcd}}}
\def\deg{\mathop{\mathopt@g{deg}}\nolimits}
\def\sum{\mathop{\xEmptyAA{sum}{mathtype=op}{mathchar=2211}}}
\def\hbar{\mathord{\xEmptyAA{hbar}{mathtype=ord}{mathchar=0127}}}
\def\angle{\mathord{\xEmptyAA{angle}{mathtype=ord}{mathchar=2220}}}
\def\cong{\mathrel{\xEmptyAA{cong}{mathtype=rel}{mathchar=2245}}}
\def\bowtie{\mathrel{\xEmptyAA{bowtie}{mathtype=rel}{mathchar=22c8}}}
\def\models{\mathrel{\xEmptyAA{models}{mathtype=rel}{mathchar=22a8}}}
\def\doteq{\mathrel{\xEmptyAA{doteq}{mathtype=rel}{mathchar=2250}}}
\def\not<{\mathrel{\xEmptyAA{notinf}{mathtype=rel}{mathchar=226e}}}
\def\not>{\mathrel{\xEmptyAA{notsup}{mathtype=rel}{mathchar=226f}}}
\def\not={\mathrel{\xEmptyAA{noteq}{mathtype=rel}{mathchar=2260}}}
\def\not\leq{\mathrel{\xEmptyAA{notleq}{mathtype=rel}{mathchar=2270}}}
\def\not\geq{\mathrel{\xEmptyAA{notgeq}{mathtype=rel}{mathchar=2271}}}
\def\not\equiv{\mathrel{\xEmptyAA{notequiv}{mathtype=rel}{mathchar=2262}}}
\def\not\prec{\mathrel{\xEmptyAA{notprec}{mathtype=rel}{mathchar=2280}}}
\def\not\succ{\mathrel{\xEmptyAA{notsucc}{mathtype=rel}{mathchar=2281}}}
\def\not\sim{\mathrel{\xEmptyAA{notsim}{mathtype=rel}{mathchar=2241}}}
\def\not\preceq{\mathrel{\xEmptyAA{notpreceq}{mathtype=rel}{mathchar=22e0}}}
\def\not\succeq{\mathrel{\xEmptyAA{notsucceq}{mathtype=rel}{mathchar=22e1}}}
\def\not\simeq{\mathrel{\xEmptyAA{notsimeq}{mathtype=rel}{mathchar=2244}}}
\def\not\subset{\mathrel{\xEmptyAA{notsubset}{mathtype=rel}{mathchar=2284}}}
\def\not\supset{\mathrel{\xEmptyAA{notsupset}{mathtype=rel}{mathchar=2285}}}
\def\not\approx{\mathrel{\xEmptyAA{notapprox}{mathtype=rel}{mathchar=2249}}}
\def\not\subseteq{\mathrel{\xEmptyAA{notsubseteq}{mathtype=rel}{mathchar=2288}}}
\def\not\supseteq{\mathrel{\xEmptyAA{notsupseteq}{mathtype=rel}{mathchar=2289}}}
\def\not\cong{\mathrel{\xEmptyAA{notcong}{mathtype=rel}{mathchar=2247}}}
\def\not\sqsubseteq{\mathrel{\xEmptyAA{notsqsubseteq}{mathtype=rel}{mathchar=22E2}}}
\def\not\sqsupseteq{\mathrel{\xEmptyAA{notsqsupseteq}{mathtype=rel}{mathchar=22E3}}}
\def\not\asymp{\mathrel{\xEmptyAA{notasymp}{mathtype=rel}{mathchar=226d}}}
\def\notin{\mathrel{\xEmptyAA{notin}{mathtype=rel}{mathchar=2209}}}
% we convert the "long" arrows to not long ones, as there is no unicode equivalent
\def\longleftarrow{\leftarrow}
\def\Longleftarrow{\Leftarrow}
\def\longrightarrow{\rightarrow}
\def\Longrightarrow{\Rightarrow}
\def\longleftrightarrow{\leftrightarrow}
\def\Longleftrightarrow{\Leftrightarrow}
\def\mapsto{\mathrel{\xEmptyAA{mapsto}{mathtype=rel}{mathchar=21a6}}}
\def\longmapsto{\mapsto}
\def\hookleftarrow{\mathrel{\xEmptyAA{hookleftarrow}{mathtype=rel}{mathchar=21a9}}}
\def\hookrightarrow{\mathrel{\xEmptyAA{hookleftarrow}{mathtype=rel}{mathchar=21aa}}}
\def\rightleftharpoons{\mathrel{\xEmptyAA{rightleftharpoons}{mathtype=rel}{mathchar=21cc}}}
\def\colon{\mathpunct{\xEmptyAA{colon}{mathtype=punct}{mathchar=3a}}}
%\subsection{Redefinition of accents}
% 
%    \begin{macrocode}
%
%accents
\let\oldaccent=\accent
\def\accent{\xEmpty{textaccent}\oldaccent}
\let\oldmathaccent=\mathaccent
\def\mathaccent{\xEmpty{mathaccent}\oldmathaccent }
%plainTeX accents as characters, e.g. \~{}
\csname h@test\endcsname
\csname h@testa\endcsname
\def\`#1{{\def\h@test{}\def\h@testa{#1}\ifx\h@test\h@testa{\xEmpty{spacing}\char"12 }\else{\ifmmode\mathaccent\else\accent\fi"12 #1}\fi}}
\def\'#1{{\def\h@test{}\def\h@testa{#1}\ifx\h@test\h@testa{\xEmpty{spacing}\char"13 }\else{\ifmmode\mathaccent\else\accent\fi"13 #1}\fi}}
\def\v#1{{\def\h@test{}\def\h@testa{#1}\ifx\h@test\h@testa{\xEmpty{spacing}\char"14 }\else{\ifmmode\mathaccent\else\accent\fi"14 #1}\fi}}
\def\u#1{{\def\h@test{}\def\h@testa{#1}\ifx\h@test\h@testa{\xEmpty{spacing}\char"15 }\else{\ifmmode\mathaccent\else\accent\fi"15 #1}\fi}}
\def\=#1{{\def\h@test{}\def\h@testa{#1}\ifx\h@test\h@testa{\xEmpty{spacing}\char"16 }\else{\ifmmode\mathaccent\else\accent\fi"16 #1}\fi}}
\def\^#1{{\def\h@test{}\def\h@testa{#1}\ifx\h@test\h@testa{\xEmpty{spacing}\char"5E }\else{\ifmmode\mathaccent\else\accent\fi"5E #1}\fi}}
\def\.#1{{\def\h@test{}\def\h@testa{#1}\ifx\h@test\h@testa{\xEmpty{spacing}\char"5F }\else{\ifmmode\mathaccent\else\accent\fi"5F #1}\fi}}
\def\H#1{{\def\h@test{}\def\h@testa{#1}\ifx\h@test\h@testa{\xEmpty{spacing}\char"7D }\else{\ifmmode\mathaccent\else\accent\fi"7D #1}\fi}}
\def\~#1{{\def\h@test{}\def\h@testa{#1}\ifx\h@test\h@testa{\xEmpty{spacing}\char"7E }\else{\ifmmode\mathaccent\else\accent\fi"7E #1}\fi}}
\def\"#1{{\def\h@test{}\def\h@testa{#1}\ifx\h@test\h@testa{\xEmpty{spacing}\char"7F }\else{\ifmmode\mathaccent\else\accent\fi"7F #1}\fi}}
\let\oldcopyright=\copyright
\def\copyright{{\xEmpty{textaccent}\oldcopyright}}%    
%    \end{macrocode}
%\subsection{Entities for some special characters}
% 
% Redefinition of some LICR commands as entities.
% \label{sec:entities}
% \emph{PARTIE \`A REVOIR}
%    \begin{macrocode}
%%% From File: ltoutenc.dtx 1565
% \textunderscore (build as a rule in OT1)
% \copyright (a circle c)
% \trademark
\DeclareTextSymbolDefault{\textbackslash}{T1}
%\def\textcopyright{[entity!\#xA9!]} no more need
%\def\textasciicircum{[entity!\#x!]} no need
%\def\textasciitilde{[entity!\#x!]}  no need
%\def\textcompwordmark{[entity!\#x!]} what's that ??
\def\textunderscore{[entity!\#x5F!]} 
\def\textvisiblespace{[entity!\#x2423!]}
\def\textellipsis{\ifmmode\mathord{\xEmptyAA{textellipsis}{mathtype=ord}{mathchar=2026}}\else[entity!\#x2026!]\fi}
\def\textregistered{[entity!\#xAE!]}
\def\texttrademark {[entity!\#x2122!]}
%\def\SS{[entity!\#x!]} no need
\def\textordfeminine{[entity!\#xAA!]}
\def\textordmasculine{[entity!\#xBA!]}
\def\textquotedblright{\xEmpty{RQUOTE}}
%    \end{macrocode}
%\subsection{Commands for spacing}
%\label{sec:spacing}
%    \begin{macrocode}
\def\thinspace{[entity!\#x2009!]}
\def\enspace{[entity!\#x2002!]}
\def\enskip{[entity!\#x2002!]}
\def\quad{[entity!\#x2001!]} % quad = 1 em
\def\qquad{[entity!\#x2001!][entity!\#x2001!]} % qquad = 2 em
%    \end{macrocode}
% \section{Tagging of font attributes (family, series \& shapes) and sizes}
% \label{sec:fonts}
%  The idea is to introduce a tag containing the font name (as it appear in the dvi) together with
%  the font attribute each time a font is loaded. The tags are stored in a global box and write at 
% the end of the document.
%    \begin{macrocode}
\newsavebox{\XX@font@list}
\newcommand{\addFontTo@list}[6]{%
\sbox\XX@font@list{\unhbox\XX@font@list\protect\xEmptyAAAAAA{fontDescription}{encoding=#1}{family=#2}{series=#3}{shape=#4}{size=#5}{name=#6}%
}}
\newcommand{\FontMatchList}{\usebox{\XX@font@list}}
\def\extract@font{%
   \get@external@font%
   \newbox{\XX@font}%
   \sbox{\XX@font}{\font@name\external@font}%
   \global\expandafter\font\font@name\external@font\relax%
 \global\addFontTo@list{\f@encoding}{\f@family}{\f@series}{\f@shape}{\f@size}{\external@font}%
    \font@name \relax%
    \csname \f@encoding+\f@family\endcsname%
    \csname\curr@fontshape\endcsname%
    \relax%
  }
\AtEndDocument{\protect\xBegin{fontList}\usebox\XX@font@list\protect\xEnd{fontList}}
%    \end{macrocode}
% %\section{Cross Reference}
% \label{sec:cr}
%    \begin{macrocode}
%%% From File: ltxref.dtx L3845
% Reference and label
\let\t@gref=\ref
\def\ref#1{\xBegin{ref}\xEmptyA{params}{idref=#1}\xBegin{refMark}{\t@gref{#1}}\xEnd{refMark}\xEnd{ref}}
\let\t@gpageref=\pageref
\def\pageref#1{\xBegin{pageref}\xEmptyA{params}{idref=#1}\xBegin{pagerefMark}{\t@gpageref{#1}}\xEnd{pagerefMark}\xEnd{pageref}}
\let\t@glabel=\label
\def\label#1{\xEmptyA{label}{idlabel=#1}\t@glabel{#1}}
%    \end{macrocode}
%
%\section{Flushing, Centering, Verbatim}
% \label{sec:ltmiscen}
% \subsection{Flushing}
%    \begin{macrocode}
%%% From File: ltmiscen.dtx L3880
% \centering
% \flush*
% \ragg*
% verbatim
% Centering/flush (L3956)
% raggleft = flushright env
% raggright = flushleft env
\let\t@gcentering=\centering
\let\t@gflushleft=\flushleft
\let\t@gendflushleft=\endflushleft
\let\t@graggedright=\raggedright
\let\t@gflushright=\flushright
\let\t@gendflushright=\endflushright
\def\endt@gcentering{\xEnd{center}}
\let\t@gcenting=\centering
\def\centering{\xEmpty{centering}\t@gcentering}
\def\endt@gflushleft{\xEnd{flushleft}}
\let\t@graggedright=\raggedright
\def\raggedright{\xBegin{flushleft}\t@graggedright\aftergroup\endt@gflushleft}
\def\endt@gflushright{\xEnd{flushright}}
\let\t@graggedleft=\raggedleft
\def\raggedleft{\xBegin{flushright}\t@graggedleft\aftergroup\endt@gflushright}
%    \end{macrocode}
%
%\subsection{Verbatim}
%\label{sec:verbatim}
% In verbatim environment, spaces and line breaks must be kept.
% In the other hand, lxir require special treatement of spacing to 
% rebuild words. A solution would be to have a ``verbatim mode'' in lxir
% to change this behaviour. Here we redefined the space in order to add an 
% empty tag \verb|\xEmpty{verbatimSpace}|. We also add a tag for line break in the
% \verb|\par| redefined macro.
%    \begin{macrocode}
% Verbatim
% with \verb 
\let\t@gverb@egroup=\verb@egroup
\def\verb@egroup{\xEnd{verb}\t@gverb@egroup} 
\let\t@gverb=\verb
\def\verb{\xBegin{verb}\def\t@gspace{\xEmpty{verbatimSpace}}\t@gverb} 
% with verbatim environment:
{\catcode`\ =\active%
\gdef\@vobeyspaces{\catcode`\ \active\def {\t@gspace}}}
\begingroup \catcode `|=0 \catcode `[= 1
\catcode`]=2 \catcode `\{=12 \catcode `\}=12
\catcode`\\=12 |gdef|@xverbatim#1\end{verbatim}[#1|end[verbatim]]
|gdef|@sxverbatim#1\end{verbatim*}[#1|end[verbatim*]]
|endgroup
\def\@verbatim{\trivlist \item\relax
  \if@minipage\else\vskip\parskip\fi
  \leftskip\@totalleftmargin\rightskip\z@skip
  \parindent\z@\parfillskip\@flushglue\parskip\z@skip
  \@@par
  \@tempswafalse
  \def\t@gspace{\xEmpty{verbatimSpace}}%
  \def\par{%
    \xEmpty{verbatimLineBreak}%
    \if@tempswa
      \leavevmode \null \@@par\penalty\interlinepenalty
    \else
      \@tempswatrue
      \ifhmode\@@par\penalty\interlinepenalty\fi
    \fi}%
  \let\do\@makeother \dospecials
  \obeylines \verbatim@font \@noligs
  \hyphenchar\font\m@ne
  \everypar \expandafter{\the\everypar \unpenalty}%
}
\let\t@gverbatim=\verbatim
\def\verbatim{\xBegin{verbatim}\t@gverbatim}
\let\t@gendverbatim=\endverbatim
\def\endverbatim{\t@gendverbatim\xEnd{verbatim}}
%    \end{macrocode}
% \subsection{Math}
% \emph{A FAIRE}
%    \begin{macrocode}
%%% From File: ltmath.dtx L4033
% math
\input lxir_math
\ifverbatimmath
	\input lxir_vmath
\fi
\AtBeginDocument{
\ifverbatimmath
	\xEmptyA{verbatimmath}{value=true}
\else
	\xEmptyA{verbatimmath}{value=false}
\fi
}
%    \end{macrocode}
% \subsection{Lists: Enumerate and Itemize}
% Other list-like environments (description, ...) are defined in classes.
% Lists environment are to be post-process so as to introduce a little bit
% of hierarchy...
% \cmd{\item} is generically tagged, but each list environnement should
% be tagged. 
%    \begin{macrocode}
% Lists (L4314) 
% enumerate L4534
\let\t@genumerate=\enumerate
\def\enumerate{\t@genumerate\xBegin{list}\xEmptyA{params}{listType=enumerate}}
\let\t@gendenumerate=\endenumerate
\def\endenumerate{\xEnd{list}\xEmptyA{params}{listType=enumerate}\t@gendenumerate}
% itemize L4545
\let\t@gitemize=\itemize
\def\itemize{\t@gitemize\xBegin{list}\xEmptyA{params}{listType=itemize}\xEmptyA{width}{w=\the\linewidth}}
\let\t@genditemize=\enditemize
\def\enditemize{\xEnd{list}\xEmptyA{params}{listType=itemize}\t@genditemize}
%    \end{macrocode}
% In the case of list of list,
% LaTeX build a box containing more than one  label.
% This behaviour is ennoying for lxir because it
% doen't allow to keep the hierachical structure.
%    \begin{macrocode}
\def\@item[#1]{%
	\xEmpty{item}
  \if@noparitem
    \@donoparitem
  \else
    \if@inlabel
      \indent \par
    \fi
    \ifhmode
      \unskip\unskip \par
    \fi
    \if@newlist
      \if@nobreak
        \@nbitem
      \else
        \addpenalty\@beginparpenalty
        \addvspace\@topsep
        \addvspace{-\parskip}%
      \fi
    \else
      \addpenalty\@itempenalty
      \addvspace\itemsep
    \fi
    \global\@inlabeltrue
  \fi
  \everypar{%
    \@minipagefalse
    \global\@newlistfalse
    \if@inlabel
      \global\@inlabelfalse
      {\setbox\z@\lastbox
       \ifvoid\z@
         \kern-\itemindent
       \fi}%
      \box\@labels
      \penalty\z@
    \fi
    \if@nobreak
      \@nobreakfalse
      \clubpenalty \@M
    \else
      \clubpenalty \@clubpenalty
      \everypar{}%
    \fi}%
  \if@noitemarg
    \@noitemargfalse
    \if@nmbrlist
      \refstepcounter\@listctr
    \fi
  \fi
% lxir : we don't accumulate the item labels     
\xBegin{itemMark}#1\xEnd{itemMark}
%\sbox\@tempboxa{#1}}%
% \global\setbox\@labels\hbox{%
%    \unhbox\@labels
%    \hskip \itemindent
%    \hskip -\labelwidth
%    \hskip -\labelsep
%    \ifdim \wd\@tempboxa >\labelwidth
%      \box\@tempboxa
%    \else
%      \hbox to\labelwidth {\unhbox\@tempboxa}%
%    \fi
%    \hskip \labelsep}%
  \ignorespaces}
% description: see classes_x_.dtx
%    \end{macrocode}
%
%\section{Boxes}
%\label{sec:boxes}
% \emph{TO  BE DONE: TAGGING OF \oarg{height} and \oarg{inner-pos} see TLC 2 p. 866}
%    \begin{macrocode}
%%% From File: ltboxes.dtx
% DONE: minipage
% DONE: \fbox
% DONE: \mbox
% DONE: \framebox[width][pos]{text}
% DONE: \parbox
% DONE: \makebox[width][pos]{text}
% WARNING: \makebox(){} NOT PROCESSED!
\let\t@gmbox=\mbox
\def\mbox#1{\xBegin{mbox}\t@gmbox{#1}\xEnd{mbox}}
\let\t@gfbox=\fbox
\def\fbox#1{\xBegin{fbox}\t@gfbox{#1}\xEnd{fbox}}
% makebox:
\let\t@g@imakebox=\@imakebox
\def\@imakebox[#1][#2]#3{\setlength\@tempdima{#1}\xBegin{makebox}\xEmptyAA{params}{width=\the\@tempdima}{pos=#2}\t@g@imakebox[#1][#2]{#3}\xEnd{makebox}}
% framebox:
\let\t@g@iframebox=\@iframebox
\def\@iframebox[#1][#2]#3{\setlength\@tempdima{#1}\xBegin{framebox}\xEmptyAA{params}{width=\the\@tempdima}{pos=#2}\t@g@iframebox[#1][#2]{#3}\xEnd{framebox}}
%minipage:
\let\t@g@iiiminipage=\@iiiminipage
\def\@iiiminipage#1#2[#3]#4{\setlength\@tempdima{#4}\xBegin{minipage}\xEmptyAA{params}{pos=#1}{width=\the\@tempdima}\t@g@iiiminipage{#1}{#2}[#3]{#4}}
\let\t@gendminipage=\endminipage
\def\endminipage{\t@gendminipage\xEnd{minipage}}
% parbox
%    \end{macrocode}
% parbox use math mode, I don't know why. To avoid any problem, the strategy for tagging is more radical here.
%    \begin{macrocode}
\renewcommand\parbox[3][c]{\setlength\@tempdima{#2}%
  \xBegin{parbox}\xEmptyAA{params}{pos=#1}{width=\the\@tempdima}{#3}\xEnd{parbox}}
%    \end{macrocode}
% \section{Tabbing}
% Relevant macros  are \@startfield, \@stopfield, \@startline and \@stopline
%    \begin{macrocode}
\let\t@gtabbing=\tabbing
\gdef\tabbing{\xBegin{tabbing}\t@gtabbing}
\let\t@gendtabbing=\endtabbing
\gdef\endtabbing{\t@gendtabbing\xEnd{tabbing}}
\gdef\@startline{%
     \ifnum \@nxttabmar >\@hightab
       \@badtab
       \global\@nxttabmar \@hightab
     \fi
     \global\@curtabmar \@nxttabmar
     \global\@curtab \@curtabmar
     \global\setbox\@curline \hbox {\xBegin{tabLine}}%
     \@startfield
     \strut}
\gdef\@stopline{%
  \unskip
  \@stopfield
  \if@rjfield
    \global\@rjfieldfalse
    \@tempdima\@totalleftmargin
    \advance\@tempdima\linewidth
    \hb@xt@\@tempdima{%
      \@itemfudge\hskip\dimen\@curtabmar
      \box\@curline
      \hfil
      \box\@curfield}%
  \else
    \@addfield
    \hbox{\@itemfudge\hskip\dimen\@curtabmar\box\@curline}%
  \fi\xEnd{tabLine}}
\gdef\@startfield{%
  \global\setbox\@curfield\hbox\bgroup\color@begingroup\xBegin{tabField}}
\gdef\@stopfield{%
  \xEnd{tabField}\color@endgroup\egroup}
% \@rtab:  \>      The \> command causes LaTeX to advance to the next tab stop.
% \@ltab:  \<      The \< command allows you to put something to the left of the local margin without changing the margin. This command can only be used at the start of the line.
% \@settab: \=   The \= command sets a tab stop at the current position.
% \@tabplus: \+  The \+ command moves the left margin of the next and all the following commands one tab stop to the right.
% \@tabminus: \- The \- command moves the left margin of the next and all the following commands one tab stop to the left.
% \@tabrj: \` The `\` command allows you to put text flushed right against any tab stop, including tab stop 0. 
% \section{Tabulation}
% Because tabular material is processed in math mode, a "math" node child is add 
% and should be suppressed in the post-process.\par\noindent
% The \cmd{\hrule} and \cmd{\vrule} are not tagged, as it's tend to have strange side effects
% too hard  to understand...
% They appear in the XML as "rule" nodes, and the a and b dimensions are sufficient to determine
% wether it's a horizontal or vertical line.
% \par\noindent
% End of Line tagging is done by redefining  \cmd{\@tabularcr} and \cmd{\@arraycr}. The resulting tag 
% is in the last ``cell'' unless a \cmd{\multicolumn} is the last element of the line: in the case, the eol 
% tag is after the multicolumn node.  
% To tag the table columns description, we create and use a  \cmd{\t@g@coldesc} box.
% Array should have the same structure...
% \emph{BUG : spaces are dropped in column separators.}\\ 
%    \begin{macrocode}
%%% From File: lttab.dtx
% Tabutation L4849
% NOT DONE : \extracolsep
\newif\if@multicol
%    \end{macrocode}
%    \begin{macrocode}
\newsavebox{\t@g@coldesc}
\let\t@garray=\array
\def\array{\xBegin{array}\t@garray}
\def\tabular{\xBegin{tabular}\let\@halignto\@empty\@tabular}
\def\endarray{\crcr\egroup\egroup\xEnd{array}} 
\def\endtabular{\crcr\egroup\egroup $\egroup\xEnd{tabular}} %$ for emacs
\def\@arraycr{%
  \xEmpty{eol}${\ifnum0=`}\fi\@ifstar\@xarraycr\@xarraycr} % $
\def\@tabularcr{%
  \xEmpty{eol}{\ifnum0=`}\fi\@ifstar\@xtabularcr\@xtabularcr}
\def\@array[#1]#2{%
        % pickup position:
        \xEmptyA{params}{pos=#1}
        % columns model tagging is done below
  \if #1t\vtop \else \if#1b\vbox \else \vcenter \fi\fi
  \bgroup
  \setbox\@arstrutbox\hbox{%
    \vrule \@height\arraystretch\ht\strutbox
           \@depth\arraystretch \dp\strutbox
           \@width\z@}%
  \@mkpream{#2}%
  \hbox{\usebox{\t@g@coldesc}}%
 \edef\@preamble{%
    \ialign \noexpand\@halignto
       \bgroup \@arstrut \@preamble \tabskip\z@skip \cr}%
  \let\@startpbox\@@startpbox \let\@endpbox\@@endpbox
  \let\tabularnewline\\%
    \let\par\@empty
    \let\@sharp##%
    \set@typeset@protect
    \lineskip\z@skip\baselineskip\z@skip
    \ifhmode \@preamerr\z@ \@@par\fi
    \@preamble}
\long\def\multicolumn#1#2#3{\multispan{#1}\begingroup
  \@multicoltrue\@mkpream{#2}\@multicolfalse%
  \def\@sharp{#3}\set@typeset@protect
  \let\@startpbox\@@startpbox\let\@endpbox\@@endpbox
  \@arstrut \xBegin{multicolumn}\xEmptyA{params}{span=#1}%
\usebox{\t@g@coldesc}\@preamble\xEnd{multicolumn}\hbox{}\endgroup\ignorespaces}
\def\@cline#1-#2\@nil{%
  \omit
  \xEmptyAA{cline}{from=#1}{to=#2}
  \@multicnt#1%
  \advance\@multispan\m@ne
  \ifnum\@multicnt=\@ne\@firstofone{&\omit}\fi
  \@multicnt#2%
  \advance\@multicnt-#1%
  \advance\@multispan\@ne
  \leaders\hrule\@height\arrayrulewidth\hfill
  \cr\noalign{\vskip-\arrayrulewidth}}
\def\@addtocoldesc#1{\sbox{\t@g@coldesc}{\ignorespaces\usebox{\t@g@coldesc}#1}}
\def\@mkpream#1{\sbox{\t@g@coldesc}{}\@addtocoldesc{\xBegin{columnsModel}}%
\@addtocoldesc{\xEmptyAAAAA{params}{arraycolsep=\the\arraycolsep}{tabcolsep=\the\tabcolsep}{arrayrulewidth=\the\arrayrulewidth}{doublerulesep=\the\doublerulesep}{arraystretch=\arraystretch}}%
\@firstamptrue\@lastchclass6
  \let\@preamble\@empty
  \let\protect\@unexpandable@protect
  \let\@sharp\relax
  \let\@startpbox\relax\let\@endpbox\relax
  \@expast{#1}%
  \expandafter\@tfor \expandafter
    \@nextchar \expandafter:\expandafter=\reserved@a\do
       {\@testpach\@nextchar
    \ifcase \@chclass \@classz \or \@classi \or \@classii \or \@classiii
      \or \@classiv \or\@classv \fi\@lastchclass\@chclass}%
  \ifcase \@lastchclass \@acol
      \or \or \@preamerr \@ne\or \@preamerr \tw@\or \or \@acol \fi\@addtocoldesc{\xEnd{columnsModel}}%
}
% tagging of colums: 
%    Character     Class       Number
%    ---------     -----       ------
%        c           0           0
%        l           0           1
%        r           0           2
%
%        |           1           -
%        @           2           -
%        p           3           -
%      {@-exp}       4           -
%      {p-arg}       5           -
\let\t@g@tabclassz=\@tabclassz
\def\@tabclassz{\ifcase \@chnum \@addtocoldesc{\xEmptyA{col}{align=center}}
                     \or \@addtocoldesc{\xEmptyA{col}{align=left}}
                     \or \@addtocoldesc{\xEmptyA{col}{align=right}\fi\t@g@tabclassz}}
\def\@tabclassz{%
%    \end{macrocode}
%  For the column definition:
%    \begin{macrocode}
\ifcase \@chnum \@addtocoldesc{\xEmptyA{col}{align=center}}
                     \or\@addtocoldesc{ \xEmptyA{col}{align=left}} 
                     \or \@addtocoldesc{\xEmptyA{col}{align=right}}\fi
%    \end{macrocode}
%  and the preamble is modified in order to tag  the cells:
%    \begin{macrocode}
  \ifcase\@lastchclass
    \@acolampacol
  \or
    \@ampacol
  \or
  \or
  \or
    \@addamp
  \or
    \@acolampacol
  \or
    \@firstampfalse\@acol
  \fi
  \edef\@preamble{%
    \@preamble{%
      \protect\xBegin{cell}%
       \ifcase\@chnum
        \hfil\ignorespaces\@sharp\unskip\hfil
      \or
        \hskip1sp\ignorespaces\@sharp\unskip\hfil
      \or
        \hfil\hskip1sp\ignorespaces\@sharp\unskip
      \fi\protect\xEnd{cell}}}}
\def\@arrayclassz{
\ifcase \@chnum \@addtocoldesc{\xEmptyA{col}{align=center}}
                     \or \@addtocoldesc{\xEmptyA{col}{align=left}}
                     \or \@addtocoldesc{\xEmptyA{col}{align=right}}\fi
\ifcase \@lastchclass \@acolampacol \or \@ampacol \or
   \or \or \@addamp \or
   \@acolampacol \or \@firstampfalse \@acol \fi
\edef\@preamble{\@preamble
\protect\xBegin{cell}
  \ifcase \@chnum
     \hfil$\relax\@sharp$\hfil \or $\relax\@sharp$\hfil
    \or \hfil$\relax\@sharp$\fi\protect\xEnd{cell}}}
\let\t@g@classi=\@classi
\def\@classi{\@addtocoldesc{\xEmpty{vline}}\t@g@classi}
% @{content}: the content appears in the array,
% but the padding is modified...
% So we tag it. 
%\let\t@g@classii=\@classii
%\def\@classii{\t@g@classii}
\def\@tabclassiv{\@addtocoldesc{\xBegin{intercolumnDef}\@nextchar\xEnd{intercolumnDef}}\@addtopreamble{\protect\xBegin{intercolumn}\@nextchar\protect\xEnd{intercolumn}}}
\def\@arrayclassiv{\@addtocoldesc{\xBegin{intercolumnDef}$\@nextchar$\xEnd{intercolumnDef}}\@addtopreamble{\protect\xBegin{intercolumn}$\@nextchar$\protect\xEnd{intercolumn}}}
% p{width}
%\let\t@g@classii=\@classii
%\def\@classii{\t@g@classii}
\def\@classv{\@addtocoldesc{\xEmptyA{par}{width=\@nextchar}}\@addtopreamble{\protect\xBegin{cell}\@startpbox{\@nextchar}\ignorespaces
\@sharp\@endpbox\protect\xEnd{cell}}}
%    \end{macrocode}
% \section{Theorems}
% \label{sec:theorem}
%    \begin{macrocode}
%%% From File: ltthm.dtx 5557
%% Is it enought ?
\def\@begintheorem#1#2{\xBegin{theorem}\trivlist
   \xBegin{theoremMark}\item[\hskip \labelsep{\bfseries \xBegin{theoremType}#1\xEnd{theoremType}\ \xBegin{theoremCount}{#2}\xEnd{theoremCount}}]\xEnd{theoremMark}\itshape}
\def\@endtheorem{\endtrivlist\xEnd{theorem}}
%    \end{macrocode}
% \section{Sectionning and \cmd{\maketitle}'s Elements}
% \label{sec:section}
% The macro \cmd{\tableofcontents} is defined in classes, but the general 
% mecanism of list of someting (toc, list of figures, ...) is provided here.
%    \begin{macrocode}
%%% From File: ltsect.dtx l5597
% title, sectioning and so on L5596
\def\title#1{\gdef\@title{\xBegin{title}#1\xEnd{title}}}
\def\author#1{\gdef\@author{\xBegin{author}#1\xEnd{author}}}
\def\date#1{\gdef\@date{\xBegin{date}#1\xEnd{date}}}
\def\thanks#1{\xBegin{thanks}\xBegin{thanksMark}\footnotemark\xEnd{thanksMark}
  \xBegin{thanksText}{#1}\xEnd{thanksText}\xEnd{thanks}}
\let\t@gand=\and
\def\and{\xEmpty{and}}
%    \end{macrocode}
% For sectioning: redefinition of \@sect is fine. See dlt.tex from Hermes, slighly modified.
%    \begin{macrocode}
\def\@sect#1#2#3#4#5#6[#7]#8{%
   \xBegin{section}%
   \xBegin{sectionHeader}\xEmptyAA{params}{sectype=#1}{level=#2}%
   \ifnum #2>\c@secnumdepth
    \let\@svsec\@empty
  \else
    \refstepcounter{#1}%
    \protected@edef\@svsec{\@seccntformat{#1}\relax}%
  \fi
  \@tempskipa #5\relax
  \ifdim \@tempskipa>\z@
    \begingroup
      #6{%
        \@hangfrom{\xBegin{sectionMark}\hskip #3\relax\@svsec\xEnd{sectionMark}}%
          \interlinepenalty \@M \xBegin{sectionTitle}#8\xEnd{sectionTitle}\@@par}%
    \endgroup
    \csname #1mark\endcsname{#7}%
    \addcontentsline{toc}{#1}{%
      \ifnum #2>\c@secnumdepth \else
        \protect\numberline{\csname the#1\endcsname}%
      \fi
      #7}%
  \else
    \def\@svsechd{%
      #6{\xBegin{sectionMark}\hskip #3
      \relax%
      \@svsec\xEnd{sectionMark} \xBegin{sectionTitle}#8\xEnd{sectionTitle}}%
      \csname #1mark\endcsname{#7}%
      \addcontentsline{toc}{#1}{%
        \ifnum #2>\c@secnumdepth \else
          \protect\numberline{\csname the#1\endcsname}%
        \fi
        #7}}%
  \fi
  \@xsect{#5}\xEnd{sectionHeader}\xEnd{section}}
% TOC and other lists: see also the classes
\def\endt@gtoc{\xEnd{listOf}}
\let\t@g@starttoc=\@starttoc
\def\@starttoc#1{\xBegin{listOf}\xEmptyA{typeList}{listType=#1}\t@g@starttoc{#1}\xEnd{listOf}}
%    \end{macrocode}
% The dots in lists are in math mode! We suppress them. 
%    \begin{macrocode}
\def\dotfill{}
\def\@dottedtocline#1#2#3#4#5{}
%    \end{macrocode}
% \section{Floats}
% \label{sec:floats}
% \emph{\cmd{\@makecaption} must be defined in the classes.
% Therefore the "captionMark" and "captionText" tags must be   
% tagged in classes as well. This is done in standard classes: article, report and book.}
%
% We don't try to distinguish float on one or two columns because we
% don't care about layout.
%    \begin{macrocode}
%%% From File: ltfloat.dtx
% caption, footnote, marginpar
%
\long\def\@caption#1[#2]#3{%
  \par\xBegin{caption}
  \addcontentsline{\csname ext@#1\endcsname}{#1}%
    {\protect\numberline{\csname the#1\endcsname}{\ignorespaces #2}}%
  \begingroup
    \@parboxrestore
    \if@minipage
      \@setminipage
    \fi
    \normalsize
    \@makecaption{\xBegin{captionMark}\csname fnum@#1\endcsname\xEnd{captionMark}}{\ignorespaces \xBegin{captionText}#3\xEnd{captionText}}\par
  \endgroup\xEnd{caption}}
% float
\def\end@float{\xEnd{float}}
\def\end@dblfloat{\xEnd{float}}
\def\@xfloat #1[#2]{\xBegin{float}\xEmptyA{params}{floatType=#1}\def\@captype{#1}}
% \marginpar
\let\t@gmarginpar=\marginpar
\def\marginpar#1{\xBegin{marginpar}{#1}\xEnd{marginpar}} 
% footnote: 
% \footnote, \footnotemark{}, \footnotetext{} (to insert in a minipage a mark related to the next \footnote{} OUTSIDE the minipage : see Latex Companion 2nd edition p.110
\def\@textsuperscript#1{%
  {\xBegin{superscript}#1\xEnd{superscript}}}
\let\t@gfootnote=\footnote
\renewcommand{\footnote}[2][]{\xBegin{footnoteCall}\if!#1!%
\xEmptyA{params}{page=\thepage}\t@gfootnote{#2}\else\xEmptyAA{params}{page=\thepage}{deffnmark=#1}\t@gfootnote[#1]{#2}\fi%
\xBegin{footnoteText}#2\xEnd{footnoteText}\xEnd{footnoteCall}}%
\let\t@g@footnotetext=\@footnotetext
\let\t@g@footnotemark=\@footnotemark
%
\let\t@g@makefnmark=\@makefnmark
\def\@makefnmark{\xBegin{footnoteMark}\xEmptyA{params}{page=\thepage}{\t@g@makefnmark}\xEnd{footnoteMark}}
% use to build the text of the footnote at the bottom of the page
\long\def\@footnotetext#1{\insert\footins{%
    \xBegin{footnote}
    \xEmptyA{params}{page=\thepage}
    \reset@font\footnotesize
    \interlinepenalty\interfootnotelinepenalty
    \splittopskip\footnotesep
    \splitmaxdepth \dp\strutbox \floatingpenalty \@MM
    \hsize\columnwidth \@parboxrestore
    \protected@edef\@currentlabel{%
       \csname p@footnote\endcsname\@thefnmark
    }%
    \color@begingroup
      \@makefntext{%
        \rule\z@\footnotesep\ignorespaces\xBegin{footnoteText}#1\xEnd{footnoteText}\@finalstrut\strutbox}%
    \color@endgroup\xEnd{footnote}}}%
% for footnotes inside minipages
\long\def\@mpfootnotetext#1{%       
  \global\setbox\@mpfootins\vbox{%
\xBegin{mpfootnote}
\xEmptyA{params}{page=\thepage}
    \unvbox\@mpfootins
    %\reset@font\footnotesize
    \hsize\columnwidth
    \@parboxrestore
    \protected@edef\@currentlabel
         {\csname p@mpfootnote\endcsname\@thefnmark}%    
    \color@begingroup
      \@makefntext{%
      \xBegin{footnoteText}%
        \rule\z@\footnotesep\ignorespaces#1\@finalstrut\strutbox\xEnd{footnoteText}}%
    \color@endgroup\xEnd{mpfootnote}}}
%    \end{macrocode}
% \section{Index}
% \label{sec:index}
% \emph{NOT YET CODED}\\
%    \begin{macrocode}
% Index L6062
%%% From File: ltidxglo.dtx
% TODO
%    \end{macrocode}
% \section{Bibliography}
% \label{sec:biblio}
% End tagging of "bibitem" must be done in post-processing. 
%    \begin{macrocode}
% partialy processed in class
\let\t@g@bibitem=\@bibitem
% probably useless:
\def\@bibitem#1{\xBegin{bibitem}\t@g@bibitem{#1}\xEnd{bibitem}}
% probably usefull:
\let\t@g@lbibitem=\@lbibitem
\def\@lbibitem[#1]#2{\xBegin{bibitem}\xEmptyA{params}{key=#2}%
        \xBegin{bibitemMark}\item[\@biblabel{#1}\hfill]\xEnd{bibitemMark}\xEnd{bibitem}\if@filesw
        {\let\protect\noexpand
        \immediate
        \write\@auxout{\string\bibcite{#2}{#1}}}\fi\ignorespaces}
%\let\t@gbibliography=\bibliography
%\def\bibliography#1{\xBegin{bibliography}\t@gbibliography{#1}\xEnd{bibliography}}
%    \end{macrocode}
% For the |\cite| tagging, we keep the loop and tag each of the citations. 
%    \begin{macrocode}
% \cite tagging:
\def\@citex[#1]#2{\leavevmode
  \xBegin{citation}
  \let\@citea\@empty
  \@cite{\@for\@citeb:=#2\do
    {\@citea\def\@citea{,\penalty\@m\ }%
     \edef\@citeb{\expandafter\@firstofone\@citeb\@empty}%
     \if@filesw\immediate\write\@auxout{\string\citation{\@citeb}}\fi
     \@ifundefined{b@\@citeb}{\hbox{\reset@font\bfseries ?}%
       \G@refundefinedtrue
       \@latex@warning
         {Citation `\@citeb' on page \thepage \space undefined}}%
       {\xBegin{cite}\xEmptyA{params}{key=\@citeb}%
    \@cite@ofmt{\csname b@\@citeb\endcsname}\xEnd{cite}}}}{#1}\xEnd{citation}}
%    \end{macrocode}
% \section{Classe and Packages Loading}
% \label{sec:class}
% We implement a mechanism to tag the class and package by overloading the macros in an additional file
% \textit{package}\texttt{\_lxir.sty} or \textit{class}\texttt{\_lxir.cls}.
% If such a file exists, it will be added at the end of the class or package. 
% \par
% We also tag the name of the class and packages used in the document.
%    \begin{macrocode}
\newsavebox{\lxir@pckg@list}
\def\add@toLXirList#1{%
  \global\sbox{\lxir@pckg@list}{\unhbox\lxir@pckg@list\xEmptyA{TaggedWith}{name=#1}}%
  \LXir@Info{add to list: #1}
}
\def\add@toPkg@Cls@List#1{%
  \global\sbox{\lxir@pckg@list}{\unhbox\lxir@pckg@list\xEmptyA{ClassOrPackageUsed}{name=#1}}%
}
%%% From File: ltclass.dtx
\def\@onefilewithoptions#1[#2][#3]#4{%
  \@pushfilename
  \xdef\@currname{#1}%
  \global\let\@currext#4%
  \expandafter\let\csname\@currname.\@currext-h@@k\endcsname\@empty
  \let\CurrentOption\@empty
  \@reset@ptions
  \makeatletter
  \def\reserved@a{%
    \@ifl@aded\@currext{#1}%
      {\@if@ptions\@currext{#1}{#2}{}%
        {\@latex@error
            {Option clash for \@cls@pkg\space #1}%
            {The package #1 has already been loaded
             with options:\MessageBreak
             \space\space[\@ptionlist{#1.\@currext}]\MessageBreak
             There has now been an attempt to load it
              with options\MessageBreak
             \space\space[#2]\MessageBreak
             Adding the global options:\MessageBreak
             \space\space
                  \@ptionlist{#1.\@currext},#2\MessageBreak
             to your \noexpand\documentclass declaration may fix this.%
             \MessageBreak
             Try typing \space <return> \space to proceed.}}}%
      {\@pass@ptions\@currext{#2}{#1}%
       \global\expandafter
       \let\csname ver@\@currname.\@currext\endcsname\@empty
       \InputIfFileExists
         {\@currname.\@currext}%
% We tag class and package
         {\add@toPkg@Cls@List{\@currname.\@currext}}%
% end of We tag class and package
         {\@missingfileerror\@currname\@currext}%
% lxir mechanism for tagging
       \def\LXir@tag@file{\@currname_lxir.\@currext}
       \InputIfFileExists
         {\LXir@tag@file}%
         {\LXir@Info{`\LXir@tag@file' loaded for tagging of `\@currname'}%
           \add@toLXirList{\LXir@tag@file}}%
         {\LXir@Info{No file `\LXir@tag@file' found for tagging of `\@currname'}}%
% end of lxir mechanism for tagging
    \let\@unprocessedoptions\@@unprocessedoptions
    \csname\@currname.\@currext-h@@k\endcsname
    \expandafter\let\csname\@currname.\@currext-h@@k\endcsname
              \@undefined
    \@unprocessedoptions}
    \@ifl@ter\@currext{#1}{#3}{}%
      {\@latex@warning@no@line
         {You have requested,\on@line,
          version\MessageBreak
            `#3' of \@cls@pkg\space #1,\MessageBreak
          but only version\MessageBreak
           `\csname ver@#1.\@currext\endcsname'\MessageBreak
          is available}}%
    \ifx\@currext\@clsextension\let\LoadClass\@twoloadclasserror\fi
    \@popfilename
    \@reset@ptions}%
  \reserved@a}
%    \end{macrocode}
%    \begin{macrocode}
\AtBeginDocument{\usebox{\lxir@pckg@list}}
%    \end{macrocode}
%    \begin{macrocode}
\let\lxir@font=\font
\def\font#1#2{\global\def#1{\xEmptyA{switch-font}{#1}}}
%    \end{macrocode}
%</lxir>
